%
% File acl2017.tex

\documentclass[11pt,a4paper]{article}
\usepackage[hyperref]{acl2017}
\usepackage{times}
\usepackage{latexsym}

\usepackage{url}

%uncomment line below for author names to appear. 
%\aclfinalcopy % Uncomment this line for the final submission
%\def\aclpaperid{***} %  Enter the acl Paper ID here

\setlength\titlebox{5cm}

\newcommand\BibTeX{B{\sc ib}\TeX}

\title{Stance Detection \& Fake News \\ NLG Proposal}

\author{
  Bernice Brown \\
					  \\
					  \\\And
  Brian Chen 	      \\
					  \\
					  \\\And
  Tahya Weiss-Gibbons \\
					  \\
					  \\\And
  Nicholas Kobald \\  \\
}
\date{}
\begin{document}
\maketitle
\section{Introduction}
The term \emph{Fake News} has gained popularity following the 2016 United States presidential election and the vote for the United Kingdom to exit the European Union \cite{rose2017brexit}\cite{kucharski2016post}. Fake News refers to articles that meet poor journalistic standards, and contain incorrect or misleading information. It's suggested that these articles, and their tendency to be shared on social media had discernible effect on the events of the USA election, and Brexit \cite{allcott2017social}. \\


Determining whether or not a news article is fake is difficult.  A Stanford study shows students from middle school through college have difficulty distinguishing real news articles from advertisements \cite{wineburg2016evaluating}. As a a result, attempts have been made to automate the detection of fake or misleading news articles \cite{conroy2015automatic}\\

The purpose of this research is to apply natural language processing and machine learning techniques to analyzing the validity of news articles. In particular, we will begin by following the outline presented by Fake News Challenge \cite{fakenewschallenge}. \\

The \emph{stance} of a text is the attitude it expresses towards a particular target \cite{augenstein2016stance}. The first step of the Fake News challenge is to categorize the stance of the body against the stance of the heading of the article. The Fake News Challenge organization provides an implementation, which we will use as our baseline. 


\section{Previous Work}

\subsection{Data Collection}
Rubin et. al. showed data used to investigate rumors and deception need to have the following characteristics. There must be both truthful and deceptive news within the data set, the format must be accessible, the data must be verifiable, and there must exist data points of comparable lengths and writing styles\cite{Rubin}. \\

\cite{conroy2015automatic}

\section{Proposed Approach}


%\bibliographystyle{acl}
%\bibliography{acl2017}
\bibliography{proposal}
\bibliographystyle{acl_natbib}

\end{document}
